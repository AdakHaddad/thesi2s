\chapter{Pendahuluan}

\section{Latar Belakang}

\hspace{1cm}Perkembangan teknologi sistem embedded dan Internet of Things (IoT) telah mengubah lanskap industri teknologi informasi secara fundamental. Menurut data dari Statista (2023), jumlah perangkat IoT yang terhubung diprediksi mencapai 29.4 miliar unit pada tahun 2030, meningkat signifikan dari 15.1 miliar unit pada tahun 2020. Pertumbuhan eksponensial ini mendorong kebutuhan akan peripheral komunikasi yang efisien, handal, dan dapat diprogram melalui software. Dua protokol komunikasi yang paling dominan dalam ekosistem sistem embedded adalah I2C (Inter-Integrated Circuit) dan I2S (Inter-IC Sound).

\hspace{1cm}Protokol I2C, yang dikembangkan oleh Philips Semiconductor pada tahun 1982, telah menjadi standar de facto untuk komunikasi serial dua-kawat (SDA dan SCL) yang memungkinkan komunikasi multi-master dan multi-slave. Dalam konteks IoT dan edge computing, I2C digunakan secara luas untuk interfacing dengan berbagai sensor seperti accelerometer, gyroscope, temperature sensor, humidity sensor, dan modul GPS. Keunggulan I2C terletak pada kesederhanaan arsitekturnya yang hanya memerlukan dua jalur komunikasi, namun mampu mendukung hingga 127 perangkat pada satu bus. Protokol ini mendukung berbagai mode kecepatan: standard mode (100 kbps), fast mode (400 kbps), fast mode plus (1 Mbps), dan high-speed mode (3.4 Mbps). Penelitian oleh Chen et al. (2020) menunjukkan bahwa 78\% dari sistem IoT komersial menggunakan I2C sebagai protokol komunikasi utama untuk sensor interfacing.

\hspace{1cm}Sementara itu, I2S (Inter-IC Sound) merupakan standar antarmuka bus serial elektrik untuk transmisi data audio digital yang dikembangkan oleh Philips pada tahun 1986. Dalam era streaming audio dan smart speakers, I2S menjadi protokol krusial untuk aplikasi audio processing, voice recognition, dan multimedia systems. I2S menggunakan tiga sinyal utama: bit clock (BCLK), word select (WS/LRCLK), dan serial data (SD). Format ini mendukung berbagai sample rate (8 kHz hingga 192 kHz) dan bit depth (16-bit, 24-bit, 32-bit), menjadikannya ideal untuk aplikasi audio berkualitas tinggi. Menurut Market Research Future (2022), pasar global untuk digital audio interface diproyeksikan tumbuh dengan CAGR 12.3\% antara 2022-2028, didorong oleh meningkatnya permintaan untuk smart home devices dan wearable audio technology.

\hspace{1cm}Dalam implementasi sistem embedded modern, FPGA (Field-Programmable Gate Array) menawarkan fleksibilitas tinggi untuk mengimplementasikan berbagai protokol komunikasi dalam bentuk IP core yang dapat dikustomisasi sesuai kebutuhan aplikasi. Berbeda dengan ASIC (Application-Specific Integrated Circuit) yang memerlukan biaya development tinggi dan waktu time-to-market yang panjang, FPGA memungkinkan rapid prototyping dan iterasi desain yang cepat. Penggunaan soft processor seperti MicroBlaze RISC-V pada FPGA memungkinkan integrasi antara hardware acceleration dan software flexibility dalam satu chip, memberikan solusi yang efisien untuk sistem embedded kompleks.

\hspace{1cm}Arsitektur RISC-V yang bersifat open-source dan modular telah mengalami adopsi yang signifikan dalam industri teknologi informasi. Berbeda dengan arsitektur proprietary seperti ARM atau x86, RISC-V menawarkan Instruction Set Architecture (ISA) yang dapat dikustomisasi tanpa biaya lisensi. Menurut laporan dari RISC-V International (2023), lebih dari 3000 anggota dari 70 negara telah bergabung dalam ekosistem RISC-V, termasuk perusahaan besar seperti Google, NVIDIA, dan Western Digital. Dalam konteks embedded systems, RISC-V menawarkan beberapa keunggulan: ukuran code yang compact, konsumsi daya yang rendah, dan kemampuan untuk menambahkan custom instructions sesuai kebutuhan aplikasi spesifik.

\hspace{1cm}Integrasi peripheral I2C dan I2S dengan sistem prosesor pada FPGA memerlukan antarmuka standar yang memfasilitasi komunikasi antara prosesor dan peripheral. Protokol AXI4-Lite (Advanced eXtensible Interface) dari ARM telah menjadi standar industri untuk komunikasi on-chip dalam sistem berbasis System-on-Chip (SoC) dan FPGA. AXI4-Lite menyediakan antarmuka yang sederhana namun efisien untuk transfer data antara master (prosesor) dan slave (peripheral), dengan mekanisme handshaking yang robust dan dukungan untuk berbagai ukuran data (32-bit atau 64-bit). Protokol ini mendukung memory-mapped I/O yang memungkinkan developer untuk mengakses peripheral register menggunakan instruksi load/store standar, menyederhanakan development software driver.

\hspace{1cm}Dari perspektif software engineering, implementasi driver untuk peripheral hardware memerlukan pemahaman yang mendalam tentang memory-mapped register, interrupt handling, dan Direct Memory Access (DMA). Pengembangan bare-metal driver atau Linux device driver untuk I2C dan I2S peripheral melibatkan beberapa layer abstraksi: Hardware Abstraction Layer (HAL), peripheral driver layer, dan application programming interface (API). Penelitian oleh Anderson dan Lee (2021) menunjukkan bahwa kualitas software driver memiliki dampak signifikan terhadap reliabilitas dan performa sistem embedded, dengan bug pada driver layer menyumbang 35\% dari total system failures.

\hspace{1cm}Penelitian sebelumnya telah menunjukkan keberhasilan implementasi berbagai protokol komunikasi pada FPGA. Kumar et al. (2018) mengimplementasikan kontroler I2C menggunakan Verilog dengan fokus pada optimasi kecepatan transfer data, mencapai throughput 380 kbps pada fast mode dengan error rate di bawah 0.01\%. Zhang dan Liu (2019) mengembangkan IP core I2S untuk aplikasi audio processing pada FPGA Xilinx dengan hasil yang menunjukkan latency di bawah 50 microseconds dan mampu menangani sample rate hingga 96 kHz. Patel dan Singh (2020) mengintegrasikan multiple I2C slaves pada single FPGA platform menggunakan AXI interconnect, mendemonstrasikan scalability dan modularity dari pendekatan IP core based design.

\hspace{1cm}Namun, implementasi terintegrasi dari kedua protokol komunikasi ini dalam satu sistem dengan prosesor RISC-V dan antarmuka AXI4-Lite masih memerlukan penelitian lebih lanjut, khususnya dalam aspek: (1) optimasi resource utilization pada FPGA untuk mendukung multiple peripheral, (2) development software ecosystem termasuk driver, API, dan example applications, (3) analisis performa sistem dalam skenario concurrent I2C transactions dan I2S audio streaming, dan (4) strategi debugging dan testing untuk hardware-software co-design.

\hspace{1cm}Kebutuhan akan sistem komunikasi yang fleksibel dan dapat dikonfigurasi melalui perangkat lunak menjadi motivasi utama dalam penelitian ini. Dengan mengimplementasikan IP core I2C dan I2S yang terintegrasi dengan MicroBlaze RISC-V melalui antarmuka AXI4-Lite, diharapkan dapat memberikan solusi yang efisien untuk aplikasi embedded yang memerlukan interfacing dengan sensor I2C dan pemrosesan audio digital secara bersamaan. Penelitian ini juga akan mengembangkan software stack lengkap termasuk bare-metal driver, library functions, dan demo applications untuk memvalidasi fungsionalitas sistem. Evaluasi akan dilakukan pada aspek performa (throughput, latency), resource utilization (LUTs, flip-flops, BRAM), dan software development experience (API usability, debugging capability).

\section{Rumusan Masalah}

Berdasarkan latar belakang yang telah diuraikan, rumusan masalah dalam penelitian ini adalah:

\begin{enumerate}
	\item Bagaimana merancang modul peripheral kontroler I2C dalam bahasa Verilog yang sesuai dengan spesifikasi protokol I2C dan dapat diintegrasikan dengan SoC berbasis MicroBlaze V melalui antarmuka AXI4-Lite?
	\item Bagaimana merancang modul peripheral transmitter I2S dalam bahasa Verilog yang mendukung format audio standar dan dapat diintegrasikan dengan SoC berbasis MicroBlaze V melalui antarmuka AXI4-Lite?
	\item Bagaimana mengintegrasikan modul peripheral I2C dan I2S yang telah dirancang ke dalam SoC MicroBlaze V menggunakan Vivado block design dan AXI interconnect?
	\item Bagaimana mengembangkan program aplikasi uji dalam bahasa C yang memanfaatkan kedua peripheral (I2C dan I2S) untuk aplikasi praktis seperti pembacaan sensor melalui I2C dan streaming audio melalui I2S?
	\item Bagaimana kinerja modul peripheral I2C dan I2S yang telah diimplementasikan dalam hal fungsionalitas komunikasi, throughput data, dan utilisasi resource FPGA?
\end{enumerate}	

\section{Tujuan Penelitian}

Tujuan penelitian pada skripsi Teknik (TE, TB, TIF) adalah menentukan sasaran atau target yang ingin dicapai melalui proses penelitian. Tujuan penelitian bisa beragam sesuai dengan bidang ilmu yang dipelajari, topik penelitian, dan permasalahan yang akan dicari solusinya.

%Secara umum, tujuan penelitian pada skripsi bidang teknik adalah untuk:
%
%\begin{itemize}
%	\item Mengidentifikasi dan menganalisis masalah atau permasalahan dalam bidang \textit{engineering}.
%	\item Meningkatkan pemahaman dan wawasan tentang bidang \textit{engineering} melalui penerapan teori dan metodologi yang sesuai.
%	\item Mengembangkan solusi atau produk baru yang inovatif dan efektif dalam 
%	bidang \textit{engineering}.
%	\item Menunjukkan penerapan prinsip-prinsip keteknikan dalam solusi atau produk yang dikembangkan.
%	\item Menjelaskan implikasi dan rekomendasi dari hasil penelitian bagi bidang \textit{engineering} dan masyarakat.
%\end{itemize}
%
%
%\noindent Catatan: Tujuan penelitian dalam skripsi bidang teknik harus jelas, spesifik, terukur, dan dapat dicapai dalam jangka waktu yang ditentukan.


\textcolor{red}{Tujuan perlu disinkronkan dengan rumusan masalah. Setiap rumusan masalah boleh memiliki lebih dari 1 tujuan. Berikut ini adalah contoh tujuan dari rumusan masalah "Bagaimanakan memperbaiki efisiensi penghematan energi pada sistem pencahayaan rumah tangga melalui implementasi teknologi kontrol otomatis?"
\begin{enumerate}
	\item Memodelkan sistem pencahayaan rumah tangga;
	\item Mendesain sistem kontrol otomatis untuk sistem pencahayaan rumah tangga.
\end{enumerate}
}

\noindent Catatan: Tujuan penelitian dalam skripsi bidang teknik harus jelas, spesifik, terukur, dan dapat dicapai dalam jangka waktu yang ditentukan.

Berikut ini adalah contoh detail tujuan untuk masing-masing prodi di DTETI:

\newpage
\vspace{5mm}
\textbf{Contoh Tujuan Penelitian Skripsi Teknik Elektro:}

\begin{minipage}{0.92\textwidth}
Berikut adalah beberapa contoh tujuan penelitian yang sesuai dengan rumusan masalah 
“perbaikan efisiensi penghematan energi pada sistem pencahayaan rumah tangga 
melalui implementasi teknologi kontrol otomatis”:
\end{minipage}



%-------------------------------------------------	
\noindent\fbox{%
	\parbox{\textwidth}{%
\begin{enumerate}
\item Menganalisis tingkat efisiensi energi pada sistem pencahayaan rumah tangga 
sebelum dan setelah implementasi teknologi kontrol otomatis.
\item Mengukur pengurangan biaya listrik setelah implementasi teknologi kontrol 
otomatis pada sistem pencahayaan rumah tangga.
\item Menunjukkan bagaimana teknologi kontrol otomatis dapat memperbaiki 
efisiensi penghematan energi pada sistem pencahayaan rumah tangga.
\item Meningkatkan kenyamanan dan keamanan pengguna rumah tangga melalui 
penggunaan teknologi kontrol otomatis pada sistem pencahayaan.
\item Menjelaskan bagaimana implementasi teknologi kontrol otomatis 
mempengaruhi kualitas cahaya dan faktor-faktor lain dalam sistem pencahayaan 
rumah tangga.
\item Membandingkan efisiensi energi dan biaya pada sistem pencahayaan rumah 
tangga dengan teknologi kontrol otomatis dan sistem manual.
\item Menunjukkan implikasi dan rekomendasi dari hasil penelitian bagi rumah tangga 
dan lingkungan.
\end{enumerate}
		
	}%
}

%-------------------------------------------------	

\newpage
\vspace{5mm}
\textbf{Contoh Tujuan Penelitian Skripsi Teknik Biomedis:}

\begin{minipage}{0.92\textwidth}
Berikut adalah beberapa contoh tujuan penelitian untuk penelitian dengan tema "Bagaimana memperbaiki akurasi deteksi kanker payudara dengan menggunakan 
teknologi pemindaian ultrasonografi berbasis AI":
\end{minipage}

%-------------------------------------------------	
\noindent\fbox{%
	\parbox{\textwidth}{%
\begin{enumerate}
\item Mengidentifikasi faktor-faktor yang mempengaruhi akurasi deteksi kanker 
payudara dengan menggunakan teknologi pemindaian ultrasonografi berbasis 
AI.
\item Menilai efektivitas teknologi pemindaian ultrasonografi berbasis AI dalam 
meningkatkan akurasi deteksi kanker payudara.
\item Menentukan metode pemindaian ultrasonografi berbasis AI yang paling efektif dalam meningkatkan akurasi deteksi kanker payudara.
\item Menilai keamanan dan tolerabilitas teknologi pemindaian ultrasonografi berbasis AI dalam deteksi kanker payudara.
\item Membandingkan akurasi deteksi kanker payudara dengan teknologi pemindaian 
ultrasonografi berbasis AI dengan metode deteksi lainnya.
\item Menyediakan bukti ilmiah untuk menunjukkan bahwa teknologi pemindaian 
ultrasonografi berbasis AI dapat digunakan sebagai metode deteksi kanker 
payudara yang lebih efektif dan akurat.
\item Meningkatkan akurasi deteksi kanker payudara dengan menggunakan teknologi 
pemindaian ultrasonografi berbasis AI.
\end{enumerate}
		
	}%
}


\section{Tujuan Penelitian}

Tujuan dari penelitian ini dibagi menjadi beberapa kategori sesuai dengan aspek pengembangan sistem:

\subsection{Tujuan Perancangan Modul Peripheral dalam Verilog}
\begin{enumerate}
	\item Merancang modul peripheral kontroler I2C dalam Verilog yang sesuai dengan spesifikasi protokol I2C standar (NXP UM10204) dan mendukung mode standard (100 kbps) dan fast mode (400 kbps).
	\item Mengimplementasikan state machine lengkap dalam Verilog untuk protokol I2C yang mencakup: START condition generation, 7-bit address transmission, R/W bit handling, data byte transfer (8-bit), ACK/NACK detection and generation, repeated START, dan STOP condition generation.
	\item Merancang modul peripheral transmitter I2S dalam Verilog yang mendukung format Philips I2S standard dengan sample rate yang dapat dikonfigurasi (8 kHz, 16 kHz, 32 kHz, 44.1 kHz, 48 kHz) dan bit depth (16-bit, 24-bit, 32-bit).
	\item Mengimplementasikan clock generation logic dalam Verilog untuk menghasilkan SCL (Serial Clock) pada I2C dan BCLK (Bit Clock), LRCLK (Left-Right Clock) pada I2S dari system clock FPGA.
	\item Merancang antarmuka AXI4-Lite slave dalam Verilog untuk kedua modul peripheral dengan register map yang mencakup: control register, status register, data register, dan configuration register.
	\item Mengimplementasikan mekanisme interrupt dalam Verilog untuk notifikasi event seperti transfer complete, error condition, dan FIFO status.
\end{enumerate}

\subsection{Tujuan Integrasi SoC MicroBlaze V}
\begin{enumerate}
	\item Membangun SoC berbasis MicroBlaze V menggunakan Vivado block design yang mencakup: MicroBlaze V processor core, AXI interconnect, memory controller (untuk BRAM atau DDR), dan custom peripheral modules (I2C, I2S, UART, Timer).
	\item Mengkonfigurasi AXI interconnect untuk menghubungkan MicroBlaze V sebagai AXI master dengan peripheral modules sebagai AXI slaves pada address space yang unik.
	\item Mengintegrasikan interrupt controller untuk menangani interrupt signals dari multiple peripherals ke MicroBlaze V processor.
	\item Melakukan address mapping yang tepat untuk peripheral registers agar dapat diakses dari program C melalui memory-mapped I/O.
	\item Mengkonfigurasi clock network untuk memastikan semua komponen SoC beroperasi pada frekuensi yang sesuai dan synchronized.
\end{enumerate}

\subsection{Tujuan Pengembangan Program C untuk Aplikasi Uji}
\begin{enumerate}
	\item Mengembangkan program C untuk inisialisasi dan konfigurasi peripheral I2C termasuk setting clock frequency, slave address, dan transfer mode.
	\item Mengimplementasikan fungsi-fungsi C untuk I2C communication: write single byte, write multiple bytes, read single byte, read multiple bytes, dan combined write-read transaction.
	\item Membuat program C untuk interfacing dengan sensor I2C seperti sensor suhu (contoh: LM75, TMP102), sensor jarak (contoh: VL53L0X), atau sensor cahaya (contoh: BH1750).
	\item Mengembangkan program C untuk inisialisasi dan konfigurasi peripheral I2S termasuk setting sample rate, bit depth, dan audio format.
	\item Mengimplementasikan fungsi-fungsi C untuk I2S operation: audio buffer preparation, data streaming, dan format configuration.
	\item Membuat aplikasi demo lengkap yang menggunakan kedua peripheral secara bersamaan: membaca data sensor via I2C dan melakukan audio streaming via I2S, mendemonstrasikan concurrent peripheral operation.
	\item Mengimplementasikan interrupt service routine (ISR) dalam C untuk menangani interrupt dari kedua peripheral (I2C dan I2S) dan memproses event secara asynchronous.
\end{enumerate}

\subsection{Tujuan Verifikasi dan Validasi}
\begin{enumerate}
	\item Melakukan simulasi RTL menggunakan Vivado Simulator untuk memverifikasi fungsionalitas modul Verilog dengan testbench yang mencakup berbagai skenario komunikasi I2C dan I2S.
	\item Memverifikasi timing diagram output dari simulasi sesuai dengan spesifikasi protokol I2C (SCL, SDA transitions) dan I2S (BCLK, LRCLK, SD timing).
	\item Melakukan synthesis dan implementation pada FPGA Xilinx (Artix-7 atau Zynq-7000) dan menganalisis resource utilization report.
	\item Melakukan hardware testing pada FPGA development board dengan menjalankan program C dan memverifikasi komunikasi dengan sensor I2C dan OLED display yang sesungguhnya.
	\item Mengukur dan menganalisis kinerja sistem: I2C transaction time, I2S audio quality, interrupt latency, dan overall system performance.
	\item Melakukan debugging menggunakan Vivado ILA (Integrated Logic Analyzer) untuk observasi signal real-time pada hardware dan troubleshooting jika terjadi error.
\end{enumerate}

\section{Batasan Penelitian}

Batasan penelitian dalam pengembangan IP core I2C dan I2S ini meliputi:

\subsection{Batasan Hardware Implementation dan Verilog Design}
\begin{enumerate}
	\item Modul peripheral I2C dirancang dalam Verilog sebagai I2C master controller saja, tidak termasuk I2C slave mode atau multi-master arbitration.
	\item Kecepatan komunikasi I2C dibatasi pada standard mode (100 kbps) dan fast mode (400 kbps), tidak termasuk fast mode plus (1 Mbps) atau high-speed mode (3.4 Mbps).
	\item Modul peripheral I2S dirancang dalam Verilog sebagai transmitter mode saja, tidak termasuk receiver atau transceiver (full-duplex) mode.
	\item Format audio I2S dibatasi pada Philips I2S standard format dengan konfigurasi stereo (2 channel), tidak termasuk format lain seperti Left-Justified, Right-Justified, atau TDM.
	\item Implementasi Verilog menggunakan behavioral modeling dan synchronous design dengan single clock domain, tidak termasuk advanced techniques seperti clock domain crossing (CDC) optimization atau asynchronous FIFO.
	\item Fitur advanced seperti DMA (Direct Memory Access) controller terintegrasi tidak termasuk dalam scope, data transfer dilakukan melalui programmed I/O dari processor.
\end{enumerate}

\subsection{Batasan SoC dan Platform FPGA}
\begin{enumerate}
	\item SoC dibangun menggunakan MicroBlaze V processor core dari Xilinx, tidak termasuk processor lain seperti MicroBlaze classic, ARM Cortex-M, atau custom RISC-V implementation.
	\item Platform FPGA yang digunakan dibatasi pada keluarga Xilinx 7-series atau Zynq-7000 (Artix-7, Kintex-7, Zynq-7010/7020), tidak termasuk FPGA vendor lain atau FPGA generasi yang lebih lama.
	\item Peripheral yang diintegrasikan dalam SoC dibatasi pada: MicroBlaze V, I2C controller (custom), I2S transmitter (custom), UART Lite, AXI Timer, dan memory controller, tidak termasuk peripheral kompleks seperti Ethernet MAC atau PCIe controller.
	\item Memory system dibatasi pada on-chip BRAM (Block RAM), tidak termasuk external DDR memory controller (kecuali jika tersedia di development board yang digunakan).
\end{enumerate}

\subsection{Batasan Pengembangan Software dan Aplikasi}
\begin{enumerate}
	\item Program aplikasi uji ditulis dalam bahasa C menggunakan Xilinx Vitis IDE, tidak termasuk bahasa lain seperti C++ atau assembly optimization.
	\item Software development dilakukan untuk bare-metal environment, tidak termasuk Real-Time Operating System (FreeRTOS, Zephyr) atau embedded Linux.
	\item Aplikasi demo dibatasi pada use case yang mendemonstrasikan kedua peripheral: (1) pembacaan sensor I2C (sensor suhu/jarak/cahaya), (2) streaming audio sederhana melalui I2S, dan (3) operasi konkuren dari I2C dan I2S untuk menunjukkan multi-peripheral capability.
	\item Device driver C dikembangkan dengan direct register access (memory-mapped I/O), tidak termasuk abstraksi layer seperti HAL yang kompleks atau middleware framework.
	\item Testing dan debugging software dibatasi pada tools yang tersedia di Vitis IDE: GDB debugger, printf debugging via UART, tidak termasuk advanced profiling tools atau code coverage analysis.
\end{enumerate}

\subsection{Batasan Testing, Validasi, dan Dokumentasi}
\begin{enumerate}
	\item Pengujian hardware dilakukan pada FPGA development board di laboratorium (contoh: Nexys A7, Basys 3, Zybo Z7), tidak pada produk komersial atau industrial environment.
	\item Sensor dan peripheral eksternal yang digunakan untuk testing I2C dibatasi pada perangkat yang tersedia di laboratorium (contoh: LM75 temperature sensor, VL53L0X distance sensor, BH1750 light sensor), sedangkan untuk I2S menggunakan DAC atau audio codec yang compatible.
	\item Validasi fungsional fokus pada correctness of communication protocol untuk kedua peripheral (I2C dan I2S) dan basic performance measurement, tidak termasuk extensive testing seperti: stress testing dengan continuous operation 24/7, fault injection testing, atau environmental testing (temperature, humidity variations).
	\item Power consumption analysis, electromagnetic interference (EMI) testing, dan signal integrity analysis tidak termasuk dalam scope penelitian.
	\item Dokumentasi dibatasi pada: block diagram SoC, state machine diagram untuk I2C dan I2S peripheral, register map specification, Verilog code comments, dan user guide untuk aplikasi C, tidak termasuk comprehensive datasheet layaknya commercial IP core.
\end{enumerate}

\section{Manfaat Penelitian}

Manfaat yang diharapkan dari penelitian ini dibagi menjadi beberapa kategori:

\subsection{Manfaat Akademik dan Pembelajaran}
\begin{enumerate}
	\item Menghasilkan modul peripheral I2C dan I2S dalam Verilog yang dapat digunakan kembali (reusable) sebagai building blocks untuk SoC design pada FPGA, dapat menjadi referensi untuk mahasiswa dan peneliti lain.
	\item Memberikan studi kasus lengkap tentang perancangan peripheral dalam Verilog mulai dari spesifikasi protokol, state machine design, hingga integrasi dengan processor-based SoC.
	\item Menyediakan template implementasi antarmuka AXI4-Lite slave dalam Verilog yang dapat diadaptasi untuk peripheral lain seperti SPI, UART, CAN, atau GPIO controller.
	\item Mendemonstrasikan metodologi hardware-software co-design pada FPGA: bagaimana Verilog modules berinteraksi dengan C programs melalui memory-mapped registers.
	\item Dapat digunakan sebagai modul pembelajaran praktikum untuk mata kuliah: Sistem Digital, Desain Digital Lanjut, Embedded Systems, Computer Architecture, atau FPGA Programming.
	\item Memberikan hands-on experience dalam full development cycle: RTL design, simulation, synthesis, implementation, C programming, debugging, dan hardware validation.
	\item Menyediakan dokumentasi lengkap termasuk: timing diagrams, state machine diagrams, register specifications, dan code examples yang dapat menjadi learning resources.
\end{enumerate}

\subsection{Manfaat Praktis untuk Proyek dan Aplikasi}
\begin{enumerate}
	\item Modul Verilog yang dihasilkan dapat langsung digunakan dalam proyek-proyek embedded systems yang memerlukan I2C sensor interfacing atau I2S audio processing pada platform FPGA.
	\item Aplikasi demo (sensor reading + OLED display) dapat menjadi foundation untuk pengembangan sistem monitoring, data logging, atau IoT edge devices berbasis FPGA.
	\item SoC architecture yang dikembangkan dapat di-scale up dengan menambahkan peripheral lain atau dioptimasi untuk specific applications (smart sensors, audio processing boards, industrial controllers).
	\item Program C yang dikembangkan dapat digunakan sebagai starting point untuk aplikasi yang lebih kompleks: multi-sensor systems, data fusion algorithms, atau real-time control systems.
	\item Methodology yang digunakan (block design approach, modular peripheral design) dapat diadopsi untuk rapid prototyping dalam proyek capstone, kompetisi robotika, atau startup technology products.
\end{enumerate}

\subsection{Manfaat untuk Komunitas Open Source dan Kolaborasi}
\begin{enumerate}
	\item Source code Verilog untuk I2C dan I2S peripheral dapat di-share ke komunitas melalui platform seperti GitHub dengan lisensi open-source (MIT, Apache 2.0, atau GPL).
	\item Kontribusi ke ekosistem MicroBlaze V dan RISC-V dengan menyediakan ready-to-use peripheral modules yang compatible dengan AXI4-Lite interface standard.
	\item Testbench dan simulation files dapat membantu developers lain dalam verification process untuk peripheral designs mereka.
	\item Block design files (.bd) dan constraint files (.xdc) dapat menjadi template untuk projects dengan hardware configuration yang similar.
	\item Dokumentasi dan tutorial dapat di-publish sebagai blog posts, YouTube videos, atau technical articles, membantu komunitas maker dan FPGA enthusiasts.
	\item Dapat menjadi basis untuk collaborative projects: misalnya menambahkan peripheral lain, porting ke FPGA boards lain, atau mengembangkan RTOS support.
\end{enumerate}

\subsection{Manfaat untuk Pengembangan Skill dan Karir}
\begin{enumerate}
	\item Memberikan pengalaman praktis dalam Verilog HDL programming untuk sequential logic, state machines, dan interface protocols—skills yang highly demanded di industri semiconductor.
	\item Mengembangkan kompetensi dalam embedded C programming: register-level programming, interrupt handling, peripheral drivers—fundamental skills untuk embedded software engineers.
	\item Hands-on experience dengan Xilinx Vivado dan Vitis tools yang merupakan industry-standard tools untuk FPGA development, meningkatkan employability di companies seperti AMD/Xilinx, Intel/Altera, atau FPGA design houses.
	\item Memahami SoC architecture dan AXI protocol yang relevan untuk positions seperti: SoC Design Engineer, FPGA Engineer, Embedded Systems Engineer, atau Hardware-Software Integration Engineer.
	\item Membangun portfolio project yang concrete dan demonstrable: dapat ditunjukkan dalam interview melalui live demo, code walkthrough, atau technical presentation.
	\item Mengembangkan soft skills: technical documentation writing, problem-solving dalam debugging hardware-software issues, systematic testing and validation approach.
	\item Mempersiapkan foundation untuk advanced topics: jika melanjutkan ke graduate studies, pengalaman ini valuable untuk research dalam computer architecture, embedded systems, atau FPGA acceleration.
\end{enumerate}



\section{Sistematika Penulisan}

Sistematika penulisan berisi pembahasan apa yang akan ditulis di setiap bab. 
Sistematika pada umumnya berupa paragraf yang setiap paragraf mencerminkan 
bahasan setiap Bab. Contoh:

\noindent Bab I membahas tentang pendahuluan yang berisi latar belakang, perumusan masalah 
dan tujuan penelitian. 

\noindent Bab II berisi tentang metodologi penelitian yang terdiri dari desain penelitian, sumber data, Teknik pengumpulan data dan Teknik analisis data.

\noindent Dan seterusnya.

