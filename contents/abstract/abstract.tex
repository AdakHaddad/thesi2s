\hspace{1cm}This research focuses on designing I2C and I2S communication peripheral modules using Verilog HDL, integrated into a System-on-Chip (SoC) based on the MicroBlaze V processor core on FPGA. These two peripheral modules serve distinct but complementary functions: the I2C peripheral module is designed to interface with sensors such as temperature, distance, or light sensors, while the I2S peripheral module handles digital audio data transmission. Additional supporting peripherals including UART (for debugging) and Timer are integrated to create a complete embedded system, but the primary research focus is on the Verilog-based RTL design of the I2C and I2S peripheral modules with AXI4-Lite slave interfaces. Test applications are written in C language, compiled using Xilinx Vitis, and executed on the MicroBlaze V-based SoC to demonstrate the functionality of both I2C and I2S peripherals.

\hspace{1cm}The methodology encompasses: (1) Verilog RTL design of I2C master controller supporting standard mode (100 kbps) and fast mode (400 kbps) with complete protocol state machine implementation, (2) Verilog RTL design of I2S transmitter module supporting configurable sample rates and bit depths for audio applications, (3) AXI4-Lite slave interface implementation with memory-mapped registers for both peripheral configurations and data transfer, (4) MicroBlaze V SoC integration using Vivado block design including processor, memory controller, AXI interconnect, and the two custom peripherals (I2C and I2S), (5) C programming for test applications demonstrating I2C sensor reading and I2S audio streaming, showcasing concurrent operation of both peripherals, and (6) comprehensive verification through RTL simulation with Vivado Simulator and hardware validation on Xilinx FPGA development board.

\hspace{1cm}The research results demonstrate successful Verilog implementation of both I2C and I2S peripheral modules that are fully functional when integrated with MicroBlaze V SoC. The I2C controller achieves reliable communication with real-world sensors at both 100 kbps and 400 kbps speeds, while the I2S module successfully handles audio data streaming with configurable formats. A practical demo application demonstrates concurrent usage: reading sensor data via I2C while simultaneously streaming audio through I2S, proving the capability to utilize both peripherals together. Resource utilization shows efficient design consuming minimal FPGA resources. The C-based test programs demonstrate smooth integration between software and hardware layers through the AXI4-Lite interface. The conclusion confirms that the designed Verilog peripheral modules (I2C and I2S) meet the requirements for building functional embedded systems on MicroBlaze V platform, providing a reusable foundation for IoT and multimedia applications.

\noindent\textbf{Keywords} : MicroBlaze V, I2C, I2S, Verilog, AXI4-Lite, FPGA, SoC, Embedded C Programming