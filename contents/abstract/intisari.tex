\hspace{1cm}Penelitian ini berfokus pada perancangan modul peripheral komunikasi I2C dan I2S menggunakan bahasa Verilog HDL yang diintegrasikan ke dalam System-on-Chip (SoC) berbasis core processor MicroBlaze V pada FPGA. Kedua modul peripheral ini memiliki fungsi yang berbeda namun saling melengkapi: modul peripheral I2C dirancang untuk melakukan interfacing dengan sensor seperti sensor suhu, jarak, atau cahaya, sedangkan modul peripheral I2S menangani transmisi data audio digital. Peripheral pendukung tambahan seperti UART (untuk debugging) dan Timer diintegrasikan untuk membentuk sistem embedded yang lengkap, namun fokus utama penelitian adalah pada desain RTL berbasis Verilog untuk modul peripheral I2C dan I2S dengan antarmuka AXI4-Lite slave. Program uji ditulis dalam bahasa C, dikompilasi menggunakan Xilinx Vitis, dan dijalankan di atas SoC berbasis MicroBlaze V untuk mendemonstrasikan fungsi dari kedua peripheral I2C dan I2S.

\hspace{1cm}Metodologi penelitian mencakup: (1) desain RTL Verilog untuk kontroler I2C master yang mendukung standard mode (100 kbps) dan fast mode (400 kbps) dengan implementasi state machine protokol yang lengkap, (2) desain RTL Verilog untuk modul transmitter I2S yang mendukung sample rate dan bit depth yang dapat dikonfigurasi untuk aplikasi audio, (3) implementasi antarmuka AXI4-Lite slave dengan register yang di-memory-mapped untuk konfigurasi kedua peripheral dan transfer data, (4) integrasi SoC MicroBlaze V menggunakan Vivado block design yang mencakup processor, memory controller, AXI interconnect, dan kedua custom peripheral (I2C dan I2S), (5) pemrograman C untuk aplikasi uji yang mendemonstrasikan pembacaan sensor I2C dan streaming audio I2S, menunjukkan operasi konkuren dari kedua peripheral, dan (6) verifikasi komprehensif melalui simulasi RTL dengan Vivado Simulator dan validasi hardware pada development board FPGA Xilinx.

\hspace{1cm}Hasil penelitian menunjukkan keberhasilan implementasi Verilog untuk kedua modul peripheral I2C dan I2S yang berfungsi dengan baik ketika diintegrasikan dengan SoC MicroBlaze V. Kontroler I2C mencapai komunikasi yang handal dengan sensor di dunia nyata pada kecepatan 100 kbps maupun 400 kbps, sementara modul I2S berhasil menangani streaming data audio dengan format yang dapat dikonfigurasi. Aplikasi demo praktis mendemonstrasikan penggunaan bersamaan: membaca data sensor melalui I2C sambil secara simultan melakukan streaming audio melalui I2S, membuktikan kemampuan untuk menggunakan kedua peripheral secara bersamaan. Utilisasi resource menunjukkan desain yang efisien dengan konsumsi resource FPGA yang minimal. Program uji berbasis C mendemonstrasikan integrasi yang mulus antara layer software dan hardware melalui antarmuka AXI4-Lite. Kesimpulan menyatakan bahwa modul peripheral Verilog yang dirancang (I2C dan I2S) memenuhi persyaratan untuk membangun sistem embedded fungsional pada platform MicroBlaze V, menyediakan fondasi yang dapat digunakan kembali untuk aplikasi IoT dan multimedia.

\noindent{Kata kunci} : MicroBlaze V, I2C, I2S, Verilog, AXI4-Lite, FPGA, SoC, Pemrograman C Embedded