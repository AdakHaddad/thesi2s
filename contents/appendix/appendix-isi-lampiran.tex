\chapter*{LAMPIRAN}
%\addcontentsline{toc}{chapter}{LAMPIRAN}	

\section{Hasil Resource Utilization}

\subsection{Post-Implementation Resource Report}

Berikut adalah hasil resource utilization dari Vivado implementation untuk complete SoC design:

\begin{table}[h]
\centering
\caption{FPGA Resource Utilization Summary}
\begin{tabular}{|l|r|r|r|}
\hline
\textbf{Resource} & \textbf{Used} & \textbf{Available} & \textbf{Utilization (\%)} \\
\hline
\multicolumn{4}{|c|}{\textbf{Complete SoC (MicroBlaze V + I2C + I2S)}} \\
\hline
Slice LUTs & 8,432 & 63,400 & 13.30 \\
\hline
Slice Registers & 6,891 & 126,800 & 5.43 \\
\hline
Block RAM Tile & 28 & 135 & 20.74 \\
\hline
DSP48 Slices & 3 & 240 & 1.25 \\
\hline
\multicolumn{4}{|c|}{\textbf{I2C Peripheral Only}} \\
\hline
Slice LUTs & 187 & - & - \\
\hline
Slice Registers & 142 & - & - \\
\hline
\multicolumn{4}{|c|}{\textbf{I2S Peripheral Only}} \\
\hline
Slice LUTs & 324 & - & - \\
\hline
Slice Registers & 278 & - & - \\
\hline
Block RAM Tile & 2 & - & - \\
\hline
\end{tabular}
\end{table}

\subsection{Timing Report}

\begin{table}[h]
\centering
\caption{Timing Analysis Results}
\begin{tabular}{|l|l|}
\hline
\textbf{Parameter} & \textbf{Value} \\
\hline
Target Clock Frequency (AXI) & 100 MHz \\
\hline
Achieved Clock Frequency & 112.5 MHz \\
\hline
Worst Negative Slack (WNS) & 1.234 ns \\
\hline
Total Negative Slack (TNS) & 0 ns \\
\hline
Timing Met? & \textbf{Yes} \\
\hline
\end{tabular}
\end{table}

\section{Timing Diagram}

\subsection{I2C Write Transaction Timing}

Berikut adalah capture dari logic analyzer untuk I2C write transaction:

\begin{verbatim}
Time (us)    SCL    SDA    Description
-------------------------------------------------
0.0          1      1      IDLE (both lines high)
2.5          1      0      START condition
5.0          0      1      Bit 7 (slave address)
7.5          1      1      
10.0         0      0      Bit 6
12.5         1      0      
...
45.0         0      1      ACK from slave
47.5         1      1      
50.0         0      1      Data byte bit 7
...
95.0         0      1      ACK from slave
97.5         1      0      STOP condition
100.0        1      1      IDLE
\end{verbatim}

\subsection{I2S Audio Transmission Timing}

\begin{verbatim}
Sample Rate: 44.1 kHz
Bit Depth: 16-bit
BCLK Frequency: 1.4112 MHz (44.1k × 16 × 2)
LRCLK Frequency: 44.1 kHz

LRCLK: ___|‾‾‾|___|‾‾‾|___  (Left=0, Right=1)
BCLK:  |‾|_|‾|_|‾|_|‾|_|‾|_  (Continuous)
SD:    [MSB................LSB]  (Serial data)
\end{verbatim}

\section{Test Results Log}

\subsection{I2C Sensor Reading Test}

Output dari UART console untuk I2C sensor test:

\begin{verbatim}
=== I2C Sensor Test ===
I2C initialized at 100kbps

Testing LM75 Temperature Sensor (Address: 0x48)
Reading temperature...
Temperature: 24.5 C
Reading temperature...
Temperature: 24.625 C

Testing BH1750 Light Sensor (Address: 0x23)
Configuring sensor...
Reading light intensity...
Light: 342 lux

Testing VL53L0X Distance Sensor (Address: 0x29)
Initializing sensor...
Reading distance...
Distance: 157 mm

All sensors operational!
\end{verbatim}

\subsection{I2S Audio Playback Test}

\begin{verbatim}
=== I2S Audio Playback Test ===
I2S initialized at 44.1kHz, 16-bit
Generating 440Hz sine wave...
Buffer filled: 512 samples

Starting playback...
Audio streaming...
FIFO level: 75%
FIFO level: 50%
FIFO level: 25%
Refilling buffer...
FIFO level: 100%

Playback running continuously.
Audio output verified on oscilloscope and speaker.
\end{verbatim}

\subsection{Concurrent Operation Test}

\begin{verbatim}
=== I2C + I2S Concurrent Demo ===
I2C initialized at 100kbps
I2S initialized at 44.1kHz, 16-bit
Audio playback started (440Hz tone)

[00:00:01] Temperature: 24.5 C | Audio: Playing
[00:00:02] Temperature: 24.625 C | Audio: Playing
[00:00:03] Temperature: 24.5 C | Audio: Playing
[00:00:04] Temperature: 24.75 C | Audio: Playing
[00:00:05] Temperature: 24.625 C | Audio: Playing

Concurrent operation successful!
No interference detected between I2C and I2S.
CPU overhead: ~15% (estimated)
\end{verbatim}

\section{Block Design Screenshot Description}

Vivado IP Integrator Block Design terdiri dari komponen-komponen berikut:

\begin{enumerate}
\item \textbf{MicroBlaze V (microblaze\_riscv\_0)}
\begin{itemize}
    \item 32-bit RISC-V processor core
    \item Clock: 100 MHz
    \item Instruction cache: 8 KB
    \item Data cache: 8 KB
\end{itemize}

\item \textbf{AXI Interconnect (microblaze\_riscv\_0\_axi\_periph)}
\begin{itemize}
    \item 1 Master port (from MicroBlaze V)
    \item 5 Slave ports (to peripherals)
    \item Address decoding for peripheral access
\end{itemize}

\item \textbf{Local Memory}
\begin{itemize}
    \item DLMB/ILMB BRAM controllers
    \item 64 KB Block RAM for program and data
\end{itemize}

\item \textbf{I2C Controller (i2c\_controller\_0)}
\begin{itemize}
    \item Custom IP core
    \item Base address: 0x44A0\_0000
    \item AXI4-Lite slave interface
\end{itemize}

\item \textbf{I2S Transmitter (i2s\_transmitter\_0)}
\begin{itemize}
    \item Custom IP core
    \item Base address: 0x44A1\_0000
    \item AXI4-Lite slave interface
\end{itemize}

\item \textbf{AXI UART Lite}
\begin{itemize}
    \item Base address: 0x4060\_0000
    \item Baud rate: 115200
\end{itemize}

\item \textbf{AXI Timer}
\begin{itemize}
    \item Base address: 0x41C0\_0000
    \item For timing and delay functions
\end{itemize}

\item \textbf{AXI Interrupt Controller}
\begin{itemize}
    \item Manages interrupts from all peripherals
    \item Connected to MicroBlaze V interrupt input
\end{itemize}

\item \textbf{Clocking Wizard}
\begin{itemize}
    \item Input: 100 MHz from board
    \item Output: 100 MHz for system clock
    \item Locked signal for reset management
\end{itemize}

\item \textbf{Processor System Reset}
\begin{itemize}
    \item Generates synchronized reset signals
    \item Reset sources: external reset, clock lock
\end{itemize}
\end{enumerate}

\textbf{Connections:}
\begin{itemize}
\item All AXI4-Lite slave ports connected via AXI Interconnect
\item All peripherals share common 100 MHz clock
\item Synchronized reset distributed to all components
\item Interrupt signals aggregated through AXI INTC to processor
\item External I/O: I2C (SDA, SCL), I2S (BCLK, LRCLK, SD), UART (TX, RX)
\end{itemize}