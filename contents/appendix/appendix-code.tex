\chapter*{LAMPIRAN}

\section{Source Code Peripheral I2C}

\subsection{I2C Controller RTL (Verilog)}

\subsubsection{I2C Top Module}
\begin{lstlisting}[language=Verilog]
module i2c_controller #(
    parameter C_S_AXI_DATA_WIDTH = 32,
    parameter C_S_AXI_ADDR_WIDTH = 4
)(
    // AXI4-Lite Interface
    input  wire                                S_AXI_ACLK,
    input  wire                                S_AXI_ARESETN,
    input  wire [C_S_AXI_ADDR_WIDTH-1:0]      S_AXI_AWADDR,
    input  wire                                S_AXI_AWVALID,
    output wire                                S_AXI_AWREADY,
    input  wire [C_S_AXI_DATA_WIDTH-1:0]      S_AXI_WDATA,
    input  wire [(C_S_AXI_DATA_WIDTH/8)-1:0]  S_AXI_WSTRB,
    input  wire                                S_AXI_WVALID,
    output wire                                S_AXI_WREADY,
    output wire [1:0]                          S_AXI_BRESP,
    output wire                                S_AXI_BVALID,
    input  wire                                S_AXI_BREADY,
    input  wire [C_S_AXI_ADDR_WIDTH-1:0]      S_AXI_ARADDR,
    input  wire                                S_AXI_ARVALID,
    output wire                                S_AXI_ARREADY,
    output wire [C_S_AXI_DATA_WIDTH-1:0]      S_AXI_RDATA,
    output wire [1:0]                          S_AXI_RRESP,
    output wire                                S_AXI_RVALID,
    input  wire                                S_AXI_RREADY,
    
    // I2C Interface
    inout  wire                                I2C_SDA,
    inout  wire                                I2C_SCL,
    
    // Interrupt
    output wire                                interrupt
);

// Register addresses
localparam ADDR_CONTROL  = 4'h0;
localparam ADDR_STATUS   = 4'h4;
localparam ADDR_DATA_TX  = 4'h8;
localparam ADDR_DATA_RX  = 4'hC;

// State machine states
localparam IDLE       = 4'd0;
localparam START      = 4'd1;
localparam ADDRESS    = 4'd2;
localparam ACK_ADDR   = 4'd3;
localparam WRITE_DATA = 4'd4;
localparam ACK_WRITE  = 4'd5;
localparam READ_DATA  = 4'd6;
localparam ACK_READ   = 4'd7;
localparam STOP       = 4'd8;

// Internal signals
reg [3:0]  state, next_state;
reg [7:0]  data_tx_reg, data_rx_reg;
reg [6:0]  slave_addr;
reg        rw_bit;
reg [7:0]  bit_counter;
reg        scl_enable;
reg        sda_out;
reg        busy;

// AXI4-Lite slave logic implementation
// ... (register read/write handling)

// I2C FSM implementation
always @(posedge S_AXI_ACLK) begin
    if (!S_AXI_ARESETN)
        state <= IDLE;
    else
        state <= next_state;
end

// ... (FSM next state logic and output logic)

endmodule
\end{lstlisting}

\newpage
\subsection{I2C Driver (C Language)}

\subsubsection{I2C Driver Header File}
\begin{lstlisting}[language=C]
// i2c_driver.h
#ifndef I2C_DRIVER_H
#define I2C_DRIVER_H

#include "xil_types.h"
#include "xstatus.h"

// I2C Register Offsets
#define I2C_CONTROL_REG    0x00
#define I2C_STATUS_REG     0x04
#define I2C_DATA_TX_REG    0x08
#define I2C_DATA_RX_REG    0x0C

// Control Register Bits
#define I2C_CTRL_START     (1 << 0)
#define I2C_CTRL_STOP      (1 << 1)
#define I2C_CTRL_READ      (1 << 2)
#define I2C_CTRL_WRITE     (1 << 3)
#define I2C_CTRL_SPEED     (1 << 4)  // 0=100kbps, 1=400kbps

// Status Register Bits
#define I2C_STAT_BUSY      (1 << 0)
#define I2C_STAT_ACK       (1 << 1)
#define I2C_STAT_DONE      (1 << 2)
#define I2C_STAT_ERROR     (1 << 3)

// Function prototypes
int i2c_init(u32 base_addr, u32 speed);
int i2c_write(u32 base_addr, u8 slave_addr, u8 *data, u32 len);
int i2c_read(u32 base_addr, u8 slave_addr, u8 *data, u32 len);
int i2c_write_read(u32 base_addr, u8 slave_addr, 
                   u8 *wr_data, u32 wr_len,
                   u8 *rd_data, u32 rd_len);

#endif // I2C_DRIVER_H
\end{lstlisting}

\newpage
\subsubsection{I2C Driver Implementation}
\begin{lstlisting}[language=C]
// i2c_driver.c
#include "i2c_driver.h"
#include "xil_io.h"
#include "sleep.h"

int i2c_init(u32 base_addr, u32 speed) {
    u32 ctrl_val = 0;
    
    // Set speed mode
    if (speed == 400000)
        ctrl_val |= I2C_CTRL_SPEED;
    
    Xil_Out32(base_addr + I2C_CONTROL_REG, ctrl_val);
    return XST_SUCCESS;
}

int i2c_write(u32 base_addr, u8 slave_addr, u8 *data, u32 len) {
    u32 i;
    u32 status;
    
    // Send START condition
    Xil_Out32(base_addr + I2C_CONTROL_REG, I2C_CTRL_START);
    usleep(10);
    
    // Send slave address with write bit
    Xil_Out32(base_addr + I2C_DATA_TX_REG, (slave_addr << 1) | 0);
    Xil_Out32(base_addr + I2C_CONTROL_REG, I2C_CTRL_WRITE);
    
    // Wait for ACK
    do {
        status = Xil_In32(base_addr + I2C_STATUS_REG);
    } while (status & I2C_STAT_BUSY);
    
    if (!(status & I2C_STAT_ACK))
        return XST_FAILURE;
    
    // Write data bytes
    for (i = 0; i < len; i++) {
        Xil_Out32(base_addr + I2C_DATA_TX_REG, data[i]);
        Xil_Out32(base_addr + I2C_CONTROL_REG, I2C_CTRL_WRITE);
        
        do {
            status = Xil_In32(base_addr + I2C_STATUS_REG);
        } while (status & I2C_STAT_BUSY);
        
        if (!(status & I2C_STAT_ACK))
            return XST_FAILURE;
    }
    
    // Send STOP condition
    Xil_Out32(base_addr + I2C_CONTROL_REG, I2C_CTRL_STOP);
    
    return XST_SUCCESS;
}

int i2c_read(u32 base_addr, u8 slave_addr, u8 *data, u32 len) {
    u32 i;
    u32 status;
    
    // Send START condition
    Xil_Out32(base_addr + I2C_CONTROL_REG, I2C_CTRL_START);
    usleep(10);
    
    // Send slave address with read bit
    Xil_Out32(base_addr + I2C_DATA_TX_REG, (slave_addr << 1) | 1);
    Xil_Out32(base_addr + I2C_CONTROL_REG, I2C_CTRL_WRITE);
    
    // Wait for ACK
    do {
        status = Xil_In32(base_addr + I2C_STATUS_REG);
    } while (status & I2C_STAT_BUSY);
    
    if (!(status & I2C_STAT_ACK))
        return XST_FAILURE;
    
    // Read data bytes
    for (i = 0; i < len; i++) {
        Xil_Out32(base_addr + I2C_CONTROL_REG, I2C_CTRL_READ);
        
        do {
            status = Xil_In32(base_addr + I2C_STATUS_REG);
        } while (status & I2C_STAT_BUSY);
        
        data[i] = Xil_In32(base_addr + I2C_DATA_RX_REG) & 0xFF;
    }
    
    // Send STOP condition
    Xil_Out32(base_addr + I2C_CONTROL_REG, I2C_CTRL_STOP);
    
    return XST_SUCCESS;
}
\end{lstlisting}

\newpage
\section{Source Code Peripheral I2S}

\subsection{I2S Transmitter RTL (Verilog)}

\subsubsection{I2S Top Module}
\begin{lstlisting}[language=Verilog]
module i2s_transmitter #(
    parameter C_S_AXI_DATA_WIDTH = 32,
    parameter C_S_AXI_ADDR_WIDTH = 4,
    parameter FIFO_DEPTH = 256
)(
    // AXI4-Lite Interface
    input  wire                                S_AXI_ACLK,
    input  wire                                S_AXI_ARESETN,
    input  wire [C_S_AXI_ADDR_WIDTH-1:0]      S_AXI_AWADDR,
    input  wire                                S_AXI_AWVALID,
    output wire                                S_AXI_AWREADY,
    input  wire [C_S_AXI_DATA_WIDTH-1:0]      S_AXI_WDATA,
    input  wire [(C_S_AXI_DATA_WIDTH/8)-1:0]  S_AXI_WSTRB,
    input  wire                                S_AXI_WVALID,
    output wire                                S_AXI_WREADY,
    output wire [1:0]                          S_AXI_BRESP,
    output wire                                S_AXI_BVALID,
    input  wire                                S_AXI_BREADY,
    input  wire [C_S_AXI_ADDR_WIDTH-1:0]      S_AXI_ARADDR,
    input  wire                                S_AXI_ARVALID,
    output wire                                S_AXI_ARREADY,
    output wire [C_S_AXI_DATA_WIDTH-1:0]      S_AXI_RDATA,
    output wire [1:0]                          S_AXI_RRESP,
    output wire                                S_AXI_RVALID,
    input  wire                                S_AXI_RREADY,
    
    // I2S Interface
    output wire                                I2S_BCLK,
    output wire                                I2S_LRCLK,
    output wire                                I2S_SD,
    
    // Interrupt
    output wire                                interrupt
);

// Register addresses
localparam ADDR_CONTROL    = 4'h0;
localparam ADDR_STATUS     = 4'h4;
localparam ADDR_DATA_FIFO  = 4'h8;
localparam ADDR_CONFIG     = 4'hC;

// State machine
localparam IDLE         = 2'd0;
localparam LEFT_CHANNEL = 2'd1;
localparam RIGHT_CHANNEL= 2'd2;

reg [1:0]  state;
reg [5:0]  bit_count;
reg [31:0] left_data, right_data;
reg        lrclk_reg, bclk_reg;
reg        sd_reg;

// FIFO signals
wire [31:0] fifo_dout;
wire        fifo_empty, fifo_full;
reg         fifo_rd_en;

// Configuration registers
reg [15:0] sample_rate_div;
reg [1:0]  bit_depth_sel; // 00=16bit, 01=24bit, 10=32bit
reg        enable;

// FIFO instantiation
fifo_audio #(
    .DATA_WIDTH(32),
    .DEPTH(FIFO_DEPTH)
) audio_fifo (
    .clk(S_AXI_ACLK),
    .rst(!S_AXI_ARESETN),
    .wr_en(fifo_wr_en),
    .din(fifo_din),
    .rd_en(fifo_rd_en),
    .dout(fifo_dout),
    .empty(fifo_empty),
    .full(fifo_full)
);

// I2S generation logic
always @(posedge S_AXI_ACLK) begin
    if (!S_AXI_ARESETN) begin
        state <= IDLE;
        bclk_reg <= 0;
        lrclk_reg <= 0;
    end else if (enable) begin
        // State machine for I2S transmission
        // ... (implementation)
    end
end

assign I2S_BCLK  = bclk_reg;
assign I2S_LRCLK = lrclk_reg;
assign I2S_SD    = sd_reg;

endmodule
\end{lstlisting}

\newpage
\subsection{I2S Driver (C Language)}

\begin{lstlisting}[language=C]
// i2s_driver.h
#ifndef I2S_DRIVER_H
#define I2S_DRIVER_H

#include "xil_types.h"

#define I2S_CONTROL_REG  0x00
#define I2S_STATUS_REG   0x04
#define I2S_DATA_REG     0x08
#define I2S_CONFIG_REG   0x0C

#define I2S_CTRL_ENABLE  (1 << 0)
#define I2S_CTRL_START   (1 << 1)
#define I2S_CTRL_STOP    (1 << 2)

typedef enum {
    I2S_RATE_8K  = 0,
    I2S_RATE_16K = 1,
    I2S_RATE_32K = 2,
    I2S_RATE_44K = 3,
    I2S_RATE_48K = 4
} i2s_sample_rate_t;

typedef enum {
    I2S_BIT_16 = 0,
    I2S_BIT_24 = 1,
    I2S_BIT_32 = 2
} i2s_bit_depth_t;

int i2s_init(u32 base_addr, i2s_sample_rate_t rate, 
             i2s_bit_depth_t depth);
int i2s_start(u32 base_addr);
int i2s_stop(u32 base_addr);
int i2s_write_samples(u32 base_addr, u32 *samples, u32 count);

#endif
\end{lstlisting}

\newpage
\section{Test Application Source Code}

\subsection{Main Application - Concurrent I2C and I2S Demo}

\begin{lstlisting}[language=C]
// main.c
#include "xparameters.h"
#include "xil_printf.h"
#include "i2c_driver.h"
#include "i2s_driver.h"
#include "lm75_sensor.h"
#include "sleep.h"
#include <math.h>

#define I2C_BASE   XPAR_I2C_CONTROLLER_0_S00_AXI_BASEADDR
#define I2S_BASE   XPAR_I2S_TRANSMITTER_0_S00_AXI_BASEADDR

#define SAMPLE_RATE 44100
#define FREQUENCY   440  // A4 note
#define AMPLITUDE   30000

// Generate sine wave samples
void generate_sine_wave(u32 *buffer, u32 num_samples) {
    int i;
    for (i = 0; i < num_samples; i++) {
        float t = (float)i / SAMPLE_RATE;
        s16 sample = (s16)(AMPLITUDE * sin(2 * M_PI * FREQUENCY * t));
        // Pack left and right channels
        buffer[i] = ((u32)sample << 16) | (sample & 0xFFFF);
    }
}

int main() {
    float temperature;
    u32 audio_buffer[512];
    int status;
    
    xil_printf("=== I2C + I2S Concurrent Demo ===\r\n");
    
    // Initialize I2C
    status = i2c_init(I2C_BASE, 100000);
    if (status != XST_SUCCESS) {
        xil_printf("I2C init failed!\r\n");
        return -1;
    }
    xil_printf("I2C initialized at 100kbps\r\n");
    
    // Initialize I2S
    status = i2s_init(I2S_BASE, I2S_RATE_44K, I2S_BIT_16);
    if (status != XST_SUCCESS) {
        xil_printf("I2S init failed!\r\n");
        return -1;
    }
    xil_printf("I2S initialized at 44.1kHz, 16-bit\r\n");
    
    // Generate audio samples
    generate_sine_wave(audio_buffer, 512);
    
    // Start I2S audio playback
    i2s_start(I2S_BASE);
    xil_printf("Audio playback started (440Hz tone)\r\n");
    
    // Main loop: read sensors while audio plays
    while (1) {
        // Read temperature from LM75
        status = lm75_read_temperature(I2C_BASE, 0x48, &temperature);
        if (status == XST_SUCCESS) {
            xil_printf("Temperature: %.2f C\r\n", temperature);
        } else {
            xil_printf("Sensor read error\r\n");
        }
        
        // Refill I2S buffer
        i2s_write_samples(I2S_BASE, audio_buffer, 512);
        
        // Wait before next reading
        sleep(1);
    }
    
    return 0;
}
\end{lstlisting}

\newpage
\section{Spesifikasi Teknis Sistem}

\subsection{Register Map I2C Peripheral}

\begin{table}[h]
\centering
\caption{I2C Controller Register Map}
\begin{tabular}{|l|l|p{6cm}|}
\hline
\textbf{Offset} & \textbf{Name} & \textbf{Description} \\
\hline
0x00 & CONTROL & Control register: START, STOP, R/W command bits \\
\hline
0x04 & STATUS & Status register: BUSY, ACK, DONE, ERROR flags \\
\hline
0x08 & DATA\_TX & Transmit data register (8-bit) \\
\hline
0x0C & DATA\_RX & Receive data register (8-bit) \\
\hline
\end{tabular}
\end{table}

\subsection{Register Map I2S Peripheral}

\begin{table}[h]
\centering
\caption{I2S Transmitter Register Map}
\begin{tabular}{|l|l|p{6cm}|}
\hline
\textbf{Offset} & \textbf{Name} & \textbf{Description} \\
\hline
0x00 & CONTROL & Control register: ENABLE, START, STOP \\
\hline
0x04 & STATUS & Status register: FIFO status, busy flag \\
\hline
0x08 & DATA\_FIFO & Audio data FIFO (32-bit samples) \\
\hline
0x0C & CONFIG & Configuration: sample rate, bit depth \\
\hline
\end{tabular}
\end{table}

\subsection{Pin Assignment}

\begin{table}[h]
\centering
\caption{FPGA Pin Assignment}
\begin{tabular}{|l|l|l|}
\hline
\textbf{Signal} & \textbf{Pin} & \textbf{Description} \\
\hline
I2C\_SDA & IO\_L12P\_T1\_MRCC\_34 & I2C Serial Data \\
\hline
I2C\_SCL & IO\_L12N\_T1\_MRCC\_34 & I2C Serial Clock \\
\hline
I2S\_BCLK & IO\_L13P\_T2\_MRCC\_34 & I2S Bit Clock \\
\hline
I2S\_LRCLK & IO\_L13N\_T2\_MRCC\_34 & I2S Left/Right Clock \\
\hline
I2S\_SD & IO\_L14P\_T2\_SRCC\_34 & I2S Serial Data \\
\hline
UART\_TX & IO\_L15P\_T2\_DQS\_34 & UART Transmit \\
\hline
UART\_RX & IO\_L15N\_T2\_DQS\_34 & UART Receive \\
\hline
\end{tabular}
\end{table}